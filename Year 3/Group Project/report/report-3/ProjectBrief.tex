

\section{What is MIPS?}

MIPS is a processor with a RISC (Reduced Instruction Set Chip) architecture. In the second year of Computing and JMC degrees at Imperial there is a laboratory exercise, as a part of the Compilers course, which involves translating a subset of Modula 2, called DEC(M2), into MIPS assembler code.


\section{Program Requirements}

Imperial College currently uses a MIPS simulator called SPIM to check that the code generated by the compiler is correct. The aim of this project is to write a more user friendly simulator which is partly customized for the needs of the course. There should be a text-only console version and a graphical version.

The program must output helpful error messages if the MIPS code given to it is not syntactically correct, and it must either give output produced by a run of correct MIPS code or output useful statistics about the program.

The graphical version must contain a help facility giving the MIPS instruction set with clear explanations, and other useful debugging tools such as the ability to set breakpoints and step through the code one instruction at a time.

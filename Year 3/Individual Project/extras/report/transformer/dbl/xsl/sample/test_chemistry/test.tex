% ------------------------------------------------------------	
% Autogenerated LaTeX file for books	
% ------------------------------------------------------------	
\ifx\pdfoutput\undefined
\documentclass[,a4paper,10pt,twoside,openright,]{report}
\else
\documentclass[pdftex,,a4paper,10pt,twoside,openright,]{report}
\fi

		\makeatletter

		% define a gold colour for hyperlinks (see above)
		\usepackage{color}
		\definecolor{gold}{rgb}{0.75,0.5,0}
		\definecolor{lightgold}{rgb}{1,.9,.4}

		% Give headers a touch of class
		\newlength{\headerwidth}
		\setlength{\headerwidth}{\textwidth}
		\addtolength{\headerwidth}{\marginparwidth}
		\addtolength{\headerwidth}{2pt}
		\def\ps@custom{
			\ps@headings
			\if@twoside
			  \renewcommand{\@evenhead}{%
				\hspace{\textwidth}%
				\hspace{-\headerwidth}%
				\color{gold}
				\rule[-4pt]{\headerwidth}{0.5pt}%
				\hspace{-\headerwidth}%
				\rule[-6pt]{\headerwidth}{0.5pt}%
				\color{black}
				\hspace{-\headerwidth}%
				\parbox{\headerwidth}{\upshape\bfseries\thepage\hfill\slshape\mdseries\leftmark}%
				}
			  \renewcommand{\@oddhead}{%
				\color{gold}
				\rule[-4pt]{\headerwidth}{0.5pt}%
				\hspace{-\headerwidth}%
				\rule[-6pt]{\headerwidth}{0.5pt}%
				\color{black}
				\hspace{-\headerwidth}%
				\parbox{\headerwidth}{\slshape\rightmark\hfill\upshape\bfseries\thepage\mdseries}
				\hspace{-\textwidth}% arbitrary
				\hfil %to prevent 'overfull hbox' errors
				}
			\fi
		}

		% custom page margins
		\usepackage{anysize}
		\setlength{\headsep}{3mm}
		\addtolength{\marginparsep}{0.5cm}
		\setlength{\marginparwidth}{4.5cm}
		\newlength{\marginparinnerwidth}
		\setlength{\marginparinnerwidth}{\marginparwidth}
		\addtolength{\marginparinnerwidth}{-0.5cm}
		\marginsize{1.5cm}{6cm}{1cm}{1.5cm}

		% custom chapter headings
		\def\@makeschapterhead#1{%
		  \vspace*{50\p@}%
		  {\parindent \z@ \raggedright \normalfont
			\Huge \sffamily \bfseries #1\par\nobreak
			\interlinepenalty\@M
			\textcolor{lightgold}{\rule{5cm}{2pt}}
			\par\nobreak
			\vskip 40\p@
		  }}
		\def\@makechapterhead#1{%
		  \vspace*{50\p@}%
		  {\parindent \z@ \raggedright \normalfont
			\Huge \sffamily \bfseries #1\par\nobreak
			\interlinepenalty\@M
			\ifnum \c@secnumdepth >\m@ne
				\colorbox{lightgold}{\parbox{5cm}{\vspace{2pt}\small\sffamily\bfseries \@chapapp\space \thechapter}}
				\par\nobreak
			\fi
			\vskip 40\p@
		}}

		% custom section headings
		\renewcommand\section{\@startsection {section}{1}{\z@}%
			{-3.5ex \@plus -1ex \@minus -.2ex}%
			{2.3ex \@plus.2ex}%
			{%
				\color{gold}
				\rule[-0.3cm]{\textwidth}{1pt}
				\hspace{-\textwidth}
				\hspace{\motifadjustment}
				\rule[-0.3cm]{1pt}{0.85cm}
				\color{black}
				\hspace{0.2ex}
				\normalfont\Large\bfseries}
			}

		\makeatother
	\label{chemistry}\usepackage{ifthen}
% --------------------------------------------
% Check for PDFLaTeX/LaTeX 
% --------------------------------------------
\newif\ifpdf
\ifx\pdfoutput\undefined
\pdffalse % we are not running PDFLaTeX
\else
\pdfoutput=1 % we are running PDFLaTeX
\pdftrue
\fi
% --------------------------------------------
% Load graphicx package with pdf if needed 
% --------------------------------------------
\ifpdf
\usepackage[pdftex]{graphicx}
\pdfcompresslevel=9
\else
\usepackage{graphicx}
\fi

\makeatletter
% redefine the listoffigures and listoftables so that the name of the chapter
% is printed whenever there are figures or tables from that chapter. encourage
% pagebreak prior to the name of the chapter (discourage orphans).
\let\save@@chapter\@chapter
\let\save@@l@figure\l@figure
\let\the@l@figure@leader\relax
\def\@chapter[#1]#2{\save@@chapter[{#1}]{#2}%
\addtocontents{lof}{\protect\def\the@l@figure@leader{\protect\pagebreak[0]\protect\contentsline{chapter}{\protect\numberline{\thechapter}#1}{}{\thepage}}}%
\addtocontents{lot}{\protect\def\the@l@figure@leader{\protect\pagebreak[0]\protect\contentsline{chapter}{\protect\numberline{\thechapter}#1}{}{\thepage}}}%
}
\renewcommand*\l@figure{\the@l@figure@leader\let\the@l@figure@leader\relax\save@@l@figure}
\let\l@table\l@figure
\makeatother
% ---------------------- 
% Most Common Packages   
% ---------------------- 
\usepackage{latexsym}         
\usepackage{enumerate}         
\usepackage{fancybox}      
\usepackage{float}       
\usepackage{ragged2e}       
\usepackage{fancyvrb}         
\makeatletter\@namedef{FV@fontfamily@default}{\def\FV@FontScanPrep{}\def\FV@FontFamily{}}\makeatother
\fvset{obeytabs=true,tabsize=3}
\usepackage{parskip}         
\usepackage{rotating}         
\usepackage{subfigure}         
\usepackage{tabularx}         
\usepackage{url}         
% --------------------------------------------
% Math support                                
% --------------------------------------------
\usepackage{amsmath,amsthm, amsfonts, amssymb, amsxtra,amsopn}
%\newtheorem{thm}{Theorem}[section]
%\newtheorem{cor}[section]{Corollary}
%\newtheorem{lem}[section]{Lemma}
%\newtheorem{defn}[section]{Definition}
%\newtheorem{prop}[section]{Proposition}
%\newtheorem{ax}{Axiom}
%\newtheorem{theorem}[section]{Theorem}
%\newtheorem{corollary}{Corollary}
%\newtheorem{lemma}{Lemma}
%\newtheorem{proposition}{Proposition}
%\theoremstyle{definition}
%\newtheorem{definition}{Definition}
%\theoremstyle{remark}
%\newtheorem{rem}{Remark}
%\newtheorem*{notation}{Notation}
%\newcommand{\ntt}{\normalfont\ttfamily}
%\newcommand{\thmref}[1]{Theorem~\ref{#1}}
%\newcommand{\secref}[1]{\S\ref{#1}}
%\newcommand{\lemref}[1]{Lemma~\ref{#1}}
 \newcommand{\bysame}{\mbox{\rule{3em}{.4pt}}\,}
 \newcommand{\A}{\mathcal{A}}
 \newcommand{\B}{\mathcal{B}}
 \newcommand{\XcY}{{(X,Y)}}
 \newcommand{\SX}{{S_X}}
 \newcommand{\SY}{{S_Y}}
 \newcommand{\SXY}{{S_{X,Y}}}
 \newcommand{\SXgYy}{{S_{X|Y}(y)}}
 \newcommand{\Cw}[1]{{\hat C_#1(X|Y)}}
 \newcommand{\G}{{G(X|Y)}}
 \newcommand{\PY}{{P_{\mathcal{Y}}}}
 \newcommand{\X}{\mathcal{X}}
 \newcommand{\wt}{\widetilde}
 \newcommand{\wh}{\widehat}
 % --------------------------------------------
 %\DeclareMathOperator{\per}{per}
 \DeclareMathOperator{\cov}{cov}
 \DeclareMathOperator{\non}{non}
 \DeclareMathOperator{\cf}{cf}
 \DeclareMathOperator{\add}{add}
 \DeclareMathOperator{\Cham}{Cham}
 \DeclareMathOperator{\IM}{Im}
 \DeclareMathOperator{\esssup}{ess\,sup}
 \DeclareMathOperator{\meas}{meas}
 \DeclareMathOperator{\seg}{seg}
% --------------------------------------------
% --------------------------------------------
% Load hyperref package with pdf if needed 
% --------------------------------------------
\ifpdf
\usepackage[pdftex,bookmarksnumbered,colorlinks,backref, bookmarks, breaklinks, linktocpage,pdfpagelabels,pdfstartview=FitH,pdfpagemode=None,pdfsubject={http://db2latex.sourceforge.net},pdftitle={test-chemistry},linkcolor=gold,anchorcolor=blue,pagecolor=blue]{hyperref}
\else
\usepackage[dvips,bookmarksnumbered,colorlinks,backref, bookmarks, breaklinks, linktocpage,]{hyperref}
\fi
% --------------------------------------------
% ----------------------------------------------
% Define a new LaTeX environment (adminipage)
% ----------------------------------------------
\newenvironment{admminipage}%
{ % this code corresponds to the \begin{adminipage} command
 \begin{Sbox}%
 \begin{minipage}%
} %done
{ % this code corresponds to the \end{adminipage} command
 \end{minipage}
 \end{Sbox}
 \fbox{\TheSbox}
} %done
% ----------------------------------------------
% Define a new LaTeX length (admlength)
% ----------------------------------------------
\newlength{\admlength}
% ----------------------------------------------
% Define a new LaTeX environment (admonition)
% With 2 parameters:
% #1 The file (e.g. note.pdf)
% #2 The caption
% ----------------------------------------------
\newenvironment{admonition}[2] 
{ % this code corresponds to the \begin{admonition} command
 \hspace{0mm}\newline\hspace*\fill\newline
 \noindent
 \setlength{\fboxsep}{5pt}
 \setlength{\admlength}{\linewidth}
 \addtolength{\admlength}{-10\fboxsep}
 \addtolength{\admlength}{-10\fboxrule}
 \admminipage{\admlength}
 {\bfseries \sc\large{#2}} \newline
 \\[1mm]
 \sffamily
 \includegraphics[width=1cm]{#1}
 \addtolength{\admlength}{-1cm}
 \addtolength{\admlength}{-20pt}
 \begin{minipage}[lt]{\admlength}
 \parskip=0.5\baselineskip \advance\parskip by 0pt plus 2pt
} %done
{ % this code corresponds to the \end{admonition} command
 \vspace{5mm} 
 \end{minipage}
 \endadmminipage
 \vspace{.5em}
 \par
}
% --------------------------------------------
% Commands to manage/style/create floats      
% figures, tables, algorithms, examples, eqn  
% --------------------------------------------
 \floatstyle{ruled}
 \restylefloat{figure}
 \floatstyle{ruled}
 \restylefloat{table}
 \floatstyle{ruled}
 \newfloat{program}{ht}{lop}[section]
 \floatstyle{ruled}
 \newfloat{example}{ht}{loe}[section]
 \floatname{example}{Example}
 \floatstyle{ruled}
 \newfloat{dbequation}{ht}{loe}[section]
 \floatname{dbequation}{Equation}
 \floatstyle{boxed}
 \newfloat{algorithm}{ht}{loa}[section]
 \floatname{algorithm}{Algorithm}
\ifpdf
\DeclareGraphicsExtensions{.pdf,.png,.jpg}
\else
\DeclareGraphicsExtensions{.eps}
\fi
% --------------------------------------------
% $latex.caption.swapskip enabled for $formal.title.placement support
\newlength{\docbooktolatextempskip}
\newcommand{\captionswapskip}{\setlength{\docbooktolatextempskip}{\abovecaptionskip}\setlength{\abovecaptionskip}{\belowcaptionskip}\setlength{\belowcaptionskip}{\docbooktolatextempskip}}
% Guard against a problem with old package versions.
\makeatletter
\AtBeginDocument{
\DeclareRobustCommand\ref{\@refstar}
\DeclareRobustCommand\pageref{\@pagerefstar}
}
\makeatother
% --------------------------------------------
\makeatletter
\newcommand{\dbz}{\penalty \z@}
\newcommand{\docbooktolatexpipe}{\ensuremath{|}\dbz}
\newskip\docbooktolatexoldparskip
\newcommand{\docbooktolatexnoparskip}{\docbooktolatexoldparskip=\parskip\parskip=0pt plus 1pt}
\newcommand{\docbooktolatexrestoreparskip}{\parskip=\docbooktolatexoldparskip}
\def\cleardoublepage{\clearpage\if@twoside \ifodd\c@page\else\hbox{}\thispagestyle{empty}\newpage\if@twocolumn\hbox{}\newpage\fi\fi\fi}
\usepackage[latin1]{inputenc}

\ifx\dblatex@chaptersmark\@undefined\def\dblatex@chaptersmark#1{\markboth{\MakeUppercase{#1}}{}}\fi
\let\save@makeschapterhead\@makeschapterhead
\def\dblatex@makeschapterhead#1{\vspace*{-80pt}\save@makeschapterhead{#1}}
\def\@makeschapterhead#1{\dblatex@makeschapterhead{#1}\dblatex@chaptersmark{#1}}

			
\AtBeginDocument{\ifx\refname\@undefined\let\docbooktolatexbibname\bibname\def\docbooktolatexbibnamex{\bibname}\else\let\docbooktolatexbibname\refname\def\docbooktolatexbibnamex{\refname}\fi}
% Facilitate use of \cite with \label
\newcommand{\docbooktolatexbibaux}[2]{%
  \protected@write\@auxout{}{\string\global\string\@namedef{docbooktolatexcite@#1}{#2}}
}
% Provide support for bibliography `subsection' environments with titles
\newenvironment{docbooktolatexbibliography}[3]{
   \begingroup
   \let\save@@chapter\chapter
   \let\save@@section\section
   \let\save@@@mkboth\@mkboth
   \let\save@@bibname\bibname
   \let\save@@refname\refname
   \let\@mkboth\@gobbletwo
   \def\@tempa{#3}
   \def\@tempb{}
   \ifx\@tempa\@tempb
      \let\chapter\@gobbletwo
      \let\section\@gobbletwo
      \let\bibname\relax
   \else
      \let\chapter#2
      \let\section#2
      \let\bibname\@tempa
   \fi
   \let\refname\bibname
   \begin{thebibliography}{#1}
}{
   \end{thebibliography}
   \let\chapter\save@@chapter
   \let\section\save@@section
   \let\@mkboth\save@@@mkboth
   \let\bibname\save@@bibname
   \let\refname\save@@refname
   \endgroup
}

		
			
%\usepackage{cite}
%\renewcommand\citeleft{(}  % parentheses around list
%\renewcommand\citeright{)} % parentheses around list
\newcommand{\docbooktolatexcite}[2]{%
  \@ifundefined{docbooktolatexcite@#1}%
  {\cite{#1}}%
  {\def\@docbooktolatextemp{#2}\ifx\@docbooktolatextemp\@empty%
   \cite{\@nameuse{docbooktolatexcite@#1}}%
   \else\cite[#2]{\@nameuse{docbooktolatexcite@#1}}%
   \fi%
  }%
}
\newcommand{\docbooktolatexbackcite}[1]{%
  \ifx\Hy@backout\@undefined\else%
    \@ifundefined{docbooktolatexcite@#1}{%
      % emit warning?
    }{%
      \ifBR@verbose%
        \PackageInfo{backref}{back cite \string`#1\string' as \string`\@nameuse{docbooktolatexcite@#1}\string'}%
      \fi%
      \Hy@backout{\@nameuse{docbooktolatexcite@#1}}%
    }%
  \fi%
}

		
			
% --------------------------------------------
% A way to honour <footnoteref>s
% Blame j-devenish (at) users.sourceforge.net
% In any other LaTeX context, this would probably go into a style file.
\newcommand{\docbooktolatexusefootnoteref}[1]{\@ifundefined{@fn@label@#1}%
  {\hbox{\@textsuperscript{\normalfont ?}}%
    \@latex@warning{Footnote label `#1' was not defined}}%
  {\@nameuse{@fn@label@#1}}}
\newcommand{\docbooktolatexmakefootnoteref}[1]{%
  \protected@write\@auxout{}%
    {\global\string\@namedef{@fn@label@#1}{\@makefnmark}}%
  \@namedef{@fn@label@#1}{\hbox{\@textsuperscript{\normalfont ?}}}%
  }

		
			
% index labeling helper
\newif\ifdocbooktolatexprintindex\docbooktolatexprintindextrue
\let\dbtolatex@@theindex\theindex
\let\dbtolatex@@endtheindex\endtheindex
\def\theindex{\relax}
\def\endtheindex{\relax}
\newenvironment{dbtolatexindex}[1]
   {
\if@openright\cleardoublepage\else\clearpage\fi
\let\dbtolatex@@indexname\indexname
\def\dbtolatex@label{%
 \ifnum \c@secnumdepth >\m@ne \refstepcounter{chapter}\fi%
 \label{#1}\hypertarget{#1}{\dbtolatex@@indexname}%
 \global\docbooktolatexprintindexfalse}
\def\indexname{\ifdocbooktolatexprintindex\dbtolatex@label\else\dbtolatex@@indexname\fi}
\dbtolatex@@theindex
   }
   {
\dbtolatex@@endtheindex\let\indexname\dbtolatex@@indexname
   }

\newlength\saveparskip \newlength\saveparindent
\newlength\tempparskip \newlength\tempparindent

		
\def\docbooktolatexgobble{\expandafter\@gobble}
% Prevent multiple openings of the same aux file
% (happens when backref is used with multiple bibliography environments)
\ifx\AfterBeginDocument\undefined\let\AfterBeginDocument\AtBeginDocument\fi
\AfterBeginDocument{
   \let\latex@@starttoc\@starttoc
   \def\@starttoc#1{%
      \@ifundefined{docbooktolatex@aux#1}{%
         \global\@namedef{docbooktolatex@aux#1}{}%
         \latex@@starttoc{#1}%
      }{}
   }
}
% --------------------------------------------
% Hacks for honouring row/entry/@align
% (\hspace not effective when in paragraph mode)
% Naming convention for these macros is:
% 'docbooktolatex' 'align' {alignment-type} {position-within-entry}
% where r = right, l = left, c = centre
\newcommand{\docbooktolatex@align}[2]{\protect\ifvmode#1\else\ifx\LT@@tabarray\@undefined#2\else#1\fi\fi}
\newcommand{\docbooktolatexalignll}{\docbooktolatex@align{\raggedright}{}}
\newcommand{\docbooktolatexalignlr}{\docbooktolatex@align{}{\hspace*\fill}}
\newcommand{\docbooktolatexaligncl}{\docbooktolatex@align{\centering}{\hfill}}
\newcommand{\docbooktolatexaligncr}{\docbooktolatex@align{}{\hspace*\fill}}
\newcommand{\docbooktolatexalignrl}{\protect\ifvmode\raggedleft\else\hfill\fi}
\newcommand{\docbooktolatexalignrr}{}
\ifx\captionswapskip\@undefined\newcommand{\captionswapskip}{}\fi
\makeatother

		\makeatletter

		% provide 'cminipage' background-shaded environment
		\def\motifadjustment{-1em}
		% Grey backgrounds for some verbatim text.
		\IfFileExists{framed.sty}{
			\newenvironment{cminipage}{\begin{shaded}}{\end{shaded}}
		  \usepackage{color,framed}
		  \definecolor{shadecolor}{gray}{0.8}
		}{
			\newlength{\cminiwidth}
			\setlength{\cminiwidth}{\textwidth}
			\addtolength{\cminiwidth}{-2\marginparsep}
			\newenvironment{cminipage}{\begin{lrbox}{\@tempboxa}\begin{minipage}{\cminiwidth}}{\end{minipage}\end{lrbox}\hspace{\marginparsep}\colorbox[gray]{0.8}{\usebox{\@tempboxa}}}
		}

		% grey background for margin notes
		\newcommand{\marginnote}[2]{%
		\marginpar{%
		\colorbox{lightgold}{\begin{tabular}{l}\hline%
		\begin{parbox}{\marginparinnerwidth}{#1}\end{parbox}%
		\\ \hline%
		\end{tabular}}%
		}%
		}

		% avoid using ampersand for LaTeX alignments -- use pipe instead
		\catcode`\|=4

		% mathematics support
		\newenvironment{rcases}{%
			\let\@ifnextchar\new@ifnextchar%
			\left.%
			\def\arraystretch{1.2}%
			\array{@{}l@{\quad}l@{}}%
		}{%
			\endarray\right\}%
		}
		\newcommand{\degr}{\:\!^{\circ}\text{C}}
		\renewcommand{\~}{\;\;\;}

		% for our "highlights" template
		\newbox\frameboxbox

		\makeatother
	\title{\bfseries Physical Chemistry}
\author{James Devenish}
% --------------------------------------------
\makeindex
\makeglossary
% --------------------------------------------

\setcounter{tocdepth}{4}

\setcounter{secnumdepth}{4}
\begin{document}

\InputIfFileExists{title}{\typeout{WARNING: Using cover page title}}{\maketitle\pagestyle{custom}\thispagestyle{empty}}
\begin{center}Copyright \copyright{} 2004 James Devenish. All rights reserved.\end{center}
% -------------------------------------------------------------
% Preface 
% -------------------------------------------------------------
\chapter*{Preface}%
\label{preface1}\hypertarget{preface1}{}%

This is an example of a \LaTeX{} document that is being converted to DocBook. It demonstrates how you might customise DB2\LaTeX{} to retain the style of custom \LaTeX{} class. It does {\em{not}} demonstrate how to convert your mathematics to MathML.
\addcontentsline{toc}{chapter}{Preface}

\docbooktolatexnoparskip
\tableofcontents
\addcontentsline{toc}{chapter}{\contentsname}
\docbooktolatexrestoreparskip
% -------------------------------------------------------------
% Preface 
% -------------------------------------------------------------
\chapter*{Introduction}%
\label{preface2}\hypertarget{preface2}{}%

This is another introductory preface.
\addcontentsline{toc}{chapter}{Introduction}

% -------------------------------------------------------------
% Chapter States of Matter 
% ------------------------------------------------------------- 	
\chapter{States of Matter}
\label{c:matter}\hypertarget{c:matter}{}%

			
			\par\noindent
			\begin{lrbox}{\frameboxbox}\begin{minipage}{\linewidth}
			
		
This Chapter covers the kinetics of intermollecular interactions with a particular focus on gases.

			
			\end{minipage}\end{lrbox}\noindent\framebox{\usebox{\frameboxbox}}
			\par
			
		

% ------------------------   
% Section 
\section{Kinetic Theory of Gases}
\label{s:ktgases}\hypertarget{s:ktgases}{}%
\marginnote{July 17}



\[
	\begin{rcases}
		$Boyle's Law$ | V\propto P^{-1}\\
		$Charles' Law$ | V\propto T\\
		$Avogadro's Law$ | V\propto n
	\end{rcases}
	\text{Explicable using the Kinetic Theory of Gases}
\]
							
 The above can be combined by the Ideal Gas Equation: 
%
\begin{eqnarray}
	PV | = | nRT
\end{eqnarray}
							


\subsection{The Postulates}
\label{section2}\hypertarget{section2}{}%

\begin{itemize}
%--- Item
\item 
Large distances between very small particles,


%--- Item
\item 
Continual, rapid, random, straight-line motion,


%--- Item
\item 
Negligible inter-particle forces,


%--- Item
\item 
Elastic collisions,


%--- Item
\item 

								$E_{\overline K}\propto T$
							 (is the same for all gases).

\end{itemize}

\vspace{0.5em}\par\noindent\begin{cminipage}{\sf\textbf{Example:
Ideal Gas Law}}\\*
%
\begin{tabular}{lll}
\begin{minipage}{4cm}
\begin{eqnarray}
	m|=|0.100\text{ g}\nonumber\\
	|=|0.0001\text{ kg}\nonumber\\
	P|=|0.0928\text{ mm}\nonumber\\
	V|=|250\text{ mL}\nonumber\\
	|=|0.25\text{ L}\nonumber\\
	T|=|23\degr\nonumber\\
	|=|300\text{ K}\nonumber\\
	R|=|0.0821\text{ L mm mol}^{-1}\text{ K}^{-1}\nonumber
\end{eqnarray}
\end{minipage}
|\ |
\begin{minipage}{4cm}
\begin{eqnarray}
	PV|=|nRT\nonumber\\
	n|=|\frac{PV}{RT}\nonumber\\
	|\approx|106\text{ g/mol}\nonumber
\end{eqnarray}
\end{minipage}
\end{tabular}
								
\end{cminipage}\vspace{0.5em}\par

\subsection{Non-ideal Gases}
\label{section3}\hypertarget{section3}{}%
\marginnote{July 18}

Gases become non-ideal at: 
\begin{itemize}
%--- Item
\item 
High pressures, or


%--- Item
\item 
Low temperatures.

\end{itemize}


This is because in these circumstances, molecules occupy appreciable volume and intermolecular forces are no longer negligible. The latter means that collisions are weaker against any container, and may arise because slow moving, low temperature particles are more easily affected by intermolecular forces and because at high pressures the molecules are closer.

\vspace{0.5em}\par\noindent\begin{cminipage}{\sf\textbf{Van der Waals Equation}}\\*
This equation is important as a conceptual reminder, because it is used to account for deviations from the ideal gas law. The equation is: 
%
\begin{eqnarray}
\left(P+\frac{\alpha n^2}{V^2}\right)\left(V-\beta n\right)|=|nRT
\end{eqnarray}
								
 where the $\alpha$ accounts for intermolecular forces and the $\beta$ accounts for the volumes of molecules.
\end{cminipage}\vspace{0.5em}\par


% ------------------------   
% Section 
\section{Effusion and Diffusion}
\label{section4}\hypertarget{section4}{}%

{\textbf{{Effusion}}}\ 
is the movement of particles through a tiny opening (aperture, pore) into another container region of lower pressure. Molecules of low weight (higher speed) elements strike barrier more frequently and therefore effuse quicker.


{\textbf{{Diffusion}}}\ 
is the mixing of substances by one spreading throughout the other substance.



% ------------------------   
% Section 
\section{Intermolecular Forces}
\label{section5}\hypertarget{section5}{}%
\marginnote{July 21}

\begin{itemize}
%--- Item
\item 
are electrostatic forces (attraction);


%--- Item
\item 
are weaker than ionic or covalent bonds;


%--- Item
\item 
are directly related to fusion and vapourisation points.

\end{itemize}

See �9.8 of Kotz and Treichel for more on polar molecules and dipole moments.

% -------------------------------------------------------------
% Chapter Solutions \& Colloids 
% ------------------------------------------------------------- 	
\chapter{Solutions \& Colloids}
\label{chapter2}\hypertarget{chapter2}{}%

\vspace{0.5em}\par\noindent\begin{cminipage}{\sf\textbf{Definition}}\\*
A {\em{solute}} dissolves into a {\em{solvent}}. 
								V$_{\text{solvent}}$ $>$ V$_{\text{solute}}$
							
\end{cminipage}\vspace{0.5em}\par

For dissolution to occur: 
\begin{enumerate}
%--- Item
\item 
Solute particles must be separate from one another;


%--- Item
\item 
Solvent particles must be separate from one another;


%--- Item
\item 
The particles of both substances must mix together.

\end{enumerate}


\vspace{0.5em}\par\noindent\begin{cminipage}{\sf\textbf{Definition}}\\*
{\em{Miscible}} substances mix together without interface.
\end{cminipage}\vspace{0.5em}\par

Miscibility is determined by the intermolecular bonds types in the solvent and solute and how these compare with each other. That is, the relative strengths of all the intermolecular bonds involved.

Dissolution occurs when the intermolecular forces between the solvent and solute are similar and strong enough to compete with the forces within the solvent itself and the solute itself.

\vspace{0.5em}\par\noindent\begin{cminipage}{\sf\textbf{Ways of Expressing Concentration}}\\*\begin{tabular}{rl}
{{Molarity (\begin{math}{{M}}\end{math})}} & {{(moles solute)/(litres solution)}} \tabularnewline
{{Molality}} & {{(moles solute)/(kg solvent)}} \tabularnewline
{{Mole Fraction (\begin{math}{{X}}\end{math})}} & {{(moles solute)/(moles solvent + moles solute)}} \tabularnewline
{{\textbackslash \%w/v}} & {{(grams solute)/(100 ml solution)}} \tabularnewline
{{\textbackslash \%w/w}} & {{(grams solute)/(100 g solution)}} \tabularnewline
{{ppm}} & {{(mg of solute)/(litre of solution)}} \tabularnewline
\end{tabular}
\end{cminipage}\vspace{0.5em}\par

When diluting, more solvent is added to a solution. This increases the solution volume, decreases the concentration, but does {\em{not}} affect the amount of solute.


% ------------------------   
% Section 
\section{Factors Affecting Solubility}
\label{section6}\hypertarget{section6}{}%
\marginnote{July 26}

You should be familiar information in the following sections.

\subsection{Pressure}
\label{section7}\hypertarget{section7}{}%

For gas into liquid, the solubility increase as gas pressure increases. There is a mathematical relationship called Henry's Law, which states 
%
\begin{eqnarray}
	S_g&=&k_H\times P_g\\
	\text{where }S_g&=&\text{gas solubility}\\
	P_g&=&\text{gas pressure}\\
	k_H&=&\text{Henry's coefficient}
\end{eqnarray}
								


For an explanation of this relationship, consider the comparative rates of vaporisation and condensation which occur at the interface of the liquid and atmosphere, and what must happen to the gas to reach equilibrium.

\subsection{Temperature}
\label{section8}\hypertarget{section8}{}%

For gases into liquids, the solubility decreases as temperature increases. This is because at higher temperatures, more energy is present to overcome intermolecular forces and therefore a phase change from liquid to gas is more likely than at lower temperatures.

For solids in liquids, it is most common that solubility increases as temperature increases, though the relationship is not necessarily linear. Some compounds are {\em{less}} soluble at higher temperatures.

% -------------------------------------------------------------
% Chapter Colligative Properties 
% ------------------------------------------------------------- 	
\chapter{Colligative Properties}
\label{chapter3}\hypertarget{chapter3}{}%
\marginnote{July 26}

Colligative properties of a solution depend on {\em{how many}} solute particle there are, not {\em{what}} they are. That is, colligative properties depend on the dissociation of substances in solution.

\vspace{0.5em}\par\noindent\begin{cminipage}{\sf\textbf{How many solute particles?}}\\*\begin{tabular}{l|l|l}
{{Moles when pure}} & {{Substance}} & {{Moles of particles when in water}} \tabularnewline
 \hline\hline
{{0.01}} & {{ethanol}} & {{0.01 mol}} \tabularnewline
 \hline
{{0.01}} & {{NaCl}} & {{0.02 mol}} \tabularnewline
 \hline
{{0.01}} & {{BaCl$_\text{2}$}} & {{0.03 mol}} \tabularnewline
\end{tabular}
\end{cminipage}\vspace{0.5em}\par


% ------------------------   
% Section 
\section{van't Hoff Factor ($i$)}
\label{section9}\hypertarget{section9}{}%

The van't Hoff factor relates the number of moles of a substance to the number of particles in solution.

%
\begin{center}\begin{tabular}{lcl}
NaCl&$\quad $NaCl$\to Na^+ + Cl^-\quad$&$i=2$\\
BaCl$_2$&$\quad $BaCl$_2\to Ba_{2+}+2Cl^-\quad$&$i=3$
\end{tabular}\end{center}
						

\vspace{0.5em}\par\noindent\begin{cminipage}{\sf\textbf{Definition}}\\*
Non-volatile electrolytes ({``}non-electrolytes{''}) have a unity van't Hoff factor. Examples include alcohols, ethylene glycol and urea.
\end{cminipage}\vspace{0.5em}\par


% ------------------------   
% Section 
\section{Freezing Point Depression}
\label{section10}\hypertarget{section10}{}%

\vspace{0.5em}\par\noindent\begin{cminipage}{\sf\textbf{Molar Mass Calculations}}\\*
Some colligative properties can be used to calculate molar masses for non-volatile ($i=1$) solvents. For example:
\begin{description}

\item[{\bf{Step 1.}}]
{
Calculate $\Delta T_F$ from given information.
}

\item[{\bf{Step 2.}}]
{
Find the molality\ $\frac{\Delta T_F}{K_F}$.
}

\item[{\bf{Step 3.}}]
{
Calculate the moles (molality $\times$ given mass).
}

\item[{\bf{Step 4.}}]
{
Find the molar mass\ $M=\frac mn$.
}
\end{description}
\end{cminipage}\vspace{0.5em}\par

\end{document}


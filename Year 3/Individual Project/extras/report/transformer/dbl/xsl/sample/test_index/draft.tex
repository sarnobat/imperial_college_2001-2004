% ------------------------------------------------------------	
% Autogenerated LaTeX file for books	
% ------------------------------------------------------------	
\ifx\pdfoutput\undefined
\documentclass[,a4paper,10pt,twoside,openright,]{report}
\else
\documentclass[pdftex,,a4paper,10pt,twoside,openright,]{report}
\fi
\label{book}\usepackage{ifthen}
% --------------------------------------------
% Check for PDFLaTeX/LaTeX 
% --------------------------------------------
\newif\ifpdf
\ifx\pdfoutput\undefined
\pdffalse % we are not running PDFLaTeX
\else
\pdfoutput=1 % we are running PDFLaTeX
\pdftrue
\fi
% --------------------------------------------
% Load graphicx package with pdf if needed 
% --------------------------------------------
\ifpdf
\usepackage[pdftex]{graphicx}
\pdfcompresslevel=9
\else
\usepackage{graphicx}
\fi
\usepackage{anysize}
\marginsize{3cm}{2cm}{1.25cm}{1.25cm}

\makeatletter
% redefine the listoffigures and listoftables so that the name of the chapter
% is printed whenever there are figures or tables from that chapter. encourage
% pagebreak prior to the name of the chapter (discourage orphans).
\let\save@@chapter\@chapter
\let\save@@l@figure\l@figure
\let\the@l@figure@leader\relax
\def\@chapter[#1]#2{\save@@chapter[{#1}]{#2}%
\addtocontents{lof}{\protect\def\the@l@figure@leader{\protect\pagebreak[0]\protect\contentsline{chapter}{\protect\numberline{\thechapter}#1}{}{\thepage}}}%
\addtocontents{lot}{\protect\def\the@l@figure@leader{\protect\pagebreak[0]\protect\contentsline{chapter}{\protect\numberline{\thechapter}#1}{}{\thepage}}}%
}
\renewcommand*\l@figure{\the@l@figure@leader\let\the@l@figure@leader\relax\save@@l@figure}
\let\l@table\l@figure
\makeatother
\usepackage{fancyhdr}
\renewcommand{\headrulewidth}{0.4pt}
\renewcommand{\footrulewidth}{0.4pt}
% Safeguard against long headers.
\usepackage{truncate}
% Use an ellipsis when text would be larger than x% of the text width.
% Preserve left/right text alignment using \hfill (works for English).
\fancyhead[ol]{\truncate{0.49\textwidth}{\sl\leftmark}}
\fancyhead[er]{\truncate{0.49\textwidth}{\hfill\sl\rightmark}}
\fancyhead[el]{\truncate{0.49\textwidth}{\sl\leftmark}}
\fancyhead[or]{\truncate{0.49\textwidth}{\hfill\sl\rightmark}}
\pagestyle{fancy}
% ---------------------- 
% Most Common Packages   
% ---------------------- 
\usepackage{latexsym}         
\usepackage{enumerate}         
\usepackage{fancybox}      
\usepackage{float}       
\usepackage{ragged2e}       
\usepackage{fancyvrb}         
\makeatletter\@namedef{FV@fontfamily@default}{\def\FV@FontScanPrep{}\def\FV@FontFamily{}}\makeatother
\fvset{obeytabs=true,tabsize=3}
\makeatletter
\let\dblatex@center\center\let\dblatex@endcenter\endcenter
\def\dblatex@nolistI{\leftmargin\leftmargini\topsep\z@ \parsep\parskip \itemsep\z@}
\def\center{\let\@listi\dblatex@nolistI\@listi\dblatex@center\let\@listi\@listI\@listi}
\def\endcenter{\dblatex@endcenter}
\makeatother
\usepackage{rotating}         
\usepackage{subfigure}         
\usepackage{tabularx}         
\usepackage{url}         
% --------------------------------------------
% Math support                                
% --------------------------------------------
\usepackage{amsmath,amsthm, amsfonts, amssymb, amsxtra,amsopn}
%\newtheorem{thm}{Theorem}[section]
%\newtheorem{cor}[section]{Corollary}
%\newtheorem{lem}[section]{Lemma}
%\newtheorem{defn}[section]{Definition}
%\newtheorem{prop}[section]{Proposition}
%\newtheorem{ax}{Axiom}
%\newtheorem{theorem}[section]{Theorem}
%\newtheorem{corollary}{Corollary}
%\newtheorem{lemma}{Lemma}
%\newtheorem{proposition}{Proposition}
%\theoremstyle{definition}
%\newtheorem{definition}{Definition}
%\theoremstyle{remark}
%\newtheorem{rem}{Remark}
%\newtheorem*{notation}{Notation}
%\newcommand{\ntt}{\normalfont\ttfamily}
%\newcommand{\thmref}[1]{Theorem~\ref{#1}}
%\newcommand{\secref}[1]{\S\ref{#1}}
%\newcommand{\lemref}[1]{Lemma~\ref{#1}}
 \newcommand{\bysame}{\mbox{\rule{3em}{.4pt}}\,}
 \newcommand{\A}{\mathcal{A}}
 \newcommand{\B}{\mathcal{B}}
 \newcommand{\XcY}{{(X,Y)}}
 \newcommand{\SX}{{S_X}}
 \newcommand{\SY}{{S_Y}}
 \newcommand{\SXY}{{S_{X,Y}}}
 \newcommand{\SXgYy}{{S_{X|Y}(y)}}
 \newcommand{\Cw}[1]{{\hat C_#1(X|Y)}}
 \newcommand{\G}{{G(X|Y)}}
 \newcommand{\PY}{{P_{\mathcal{Y}}}}
 \newcommand{\X}{\mathcal{X}}
 \newcommand{\wt}{\widetilde}
 \newcommand{\wh}{\widehat}
 % --------------------------------------------
 %\DeclareMathOperator{\per}{per}
 \DeclareMathOperator{\cov}{cov}
 \DeclareMathOperator{\non}{non}
 \DeclareMathOperator{\cf}{cf}
 \DeclareMathOperator{\add}{add}
 \DeclareMathOperator{\Cham}{Cham}
 \DeclareMathOperator{\IM}{Im}
 \DeclareMathOperator{\esssup}{ess\,sup}
 \DeclareMathOperator{\meas}{meas}
 \DeclareMathOperator{\seg}{seg}
% --------------------------------------------
% ---------------
% Document Font  
% ---------------
\usepackage{mathptm,courier}
% --------------------------------------------
% Load hyperref package with pdf if needed 
% --------------------------------------------
\ifpdf
\usepackage[pdftex,bookmarksnumbered,colorlinks,backref, bookmarks, breaklinks, linktocpage,pdfpagelabels,pdfstartview=FitH,pdfpagemode=None,pdfsubject={http://db2latex.sourceforge.net}]{hyperref}
\else
\usepackage[bookmarksnumbered,colorlinks,backref, bookmarks, breaklinks, linktocpage,]{hyperref}
\fi
% --------------------------------------------
% ----------------------------------------------
% Define a new LaTeX environment (adminipage)
% ----------------------------------------------
\newenvironment{admminipage}%
{ % this code corresponds to the \begin{adminipage} command
 \begin{Sbox}%
 \begin{minipage}%
} %done
{ % this code corresponds to the \end{adminipage} command
 \end{minipage}
 \end{Sbox}
 \fbox{\TheSbox}
} %done
% ----------------------------------------------
% Define a new LaTeX length (admlength)
% ----------------------------------------------
\newlength{\admlength}
% ----------------------------------------------
% Define a new LaTeX environment (admonition)
% With 2 parameters:
% #1 The file (e.g. note.pdf)
% #2 The caption
% ----------------------------------------------
\newenvironment{admonition}[2] 
{ % this code corresponds to the \begin{admonition} command
 \hspace{0mm}\newline\hspace*\fill\newline
 \noindent
 \setlength{\fboxsep}{5pt}
 \setlength{\admlength}{\linewidth}
 \addtolength{\admlength}{-10\fboxsep}
 \addtolength{\admlength}{-10\fboxrule}
 \admminipage{\admlength}
 {\bfseries \sc\large{#2}} \newline
 \\[1mm]
 \sffamily
 \includegraphics[width=1cm]{#1}
 \addtolength{\admlength}{-1cm}
 \addtolength{\admlength}{-20pt}
 \begin{minipage}[lt]{\admlength}
 \parskip=0.5\baselineskip \advance\parskip by 0pt plus 2pt
} %done
{ % this code corresponds to the \end{admonition} command
 \vspace{5mm} 
 \end{minipage}
 \endadmminipage
 \vspace{.5em}
 \par
}
% --------------------------------------------
% Commands to manage/style/create floats      
% figures, tables, algorithms, examples, eqn  
% --------------------------------------------
 \floatstyle{ruled}
 \restylefloat{figure}
 \floatstyle{ruled}
 \restylefloat{table}
 \floatstyle{ruled}
 \newfloat{program}{ht}{lop}[section]
 \floatstyle{ruled}
 \newfloat{example}{ht}{loe}[section]
 \floatname{example}{Example}
 \floatstyle{ruled}
 \newfloat{dbequation}{ht}{loe}[section]
 \floatname{dbequation}{Equation}
 \floatstyle{boxed}
 \newfloat{algorithm}{ht}{loa}[section]
 \floatname{algorithm}{Algorithm}
\ifpdf
\DeclareGraphicsExtensions{.pdf,.png,.jpg}
\else
\DeclareGraphicsExtensions{.eps}
\fi
% --------------------------------------------
% $latex.caption.swapskip enabled for $formal.title.placement support
\newlength{\docbooktolatextempskip}
\newcommand{\captionswapskip}{\setlength{\docbooktolatextempskip}{\abovecaptionskip}\setlength{\abovecaptionskip}{\belowcaptionskip}\setlength{\belowcaptionskip}{\docbooktolatextempskip}}
% Guard against a problem with old package versions.
\makeatletter
\AtBeginDocument{
\DeclareRobustCommand\ref{\@refstar}
\DeclareRobustCommand\pageref{\@pagerefstar}
}
\makeatother
\usepackage[color]{showkeys}
\definecolor{refkey}{gray}{0.5}
\definecolor{labelkey}{gray}{0.5}
% Rip off things from showkeys to highlight index references
\definecolor{indexkey}{gray}{.5}%
\makeatletter
\def\SK@indexcolor{\color{indexkey}}
\def\SK@@@index#1{\@bsphack\SK@\SK@@index{#1}\begingroup\SK@index{#1}\endgroup\@esphack}
\def\SK@@index#1>#2\SK@{\leavevmode\vbox to\z@{\vss \SK@indexcolor \rlap{\vrule\raise .75em\hbox{}{\circle*{5}}}}}
\AtBeginDocument{\let\SK@index\index
\let\index\SK@@@index}
\makeatother
% --------------------------------------------
\makeatletter
\newcommand{\dbz}{\penalty \z@}
\newcommand{\docbooktolatexpipe}{\ensuremath{|}\dbz}
\newskip\docbooktolatexoldparskip
\newcommand{\docbooktolatexnoparskip}{\docbooktolatexoldparskip=\parskip\parskip=0pt plus 1pt}
\newcommand{\docbooktolatexrestoreparskip}{\parskip=\docbooktolatexoldparskip}
\def\cleardoublepage{\clearpage\if@twoside \ifodd\c@page\else\hbox{}\thispagestyle{empty}\newpage\if@twocolumn\hbox{}\newpage\fi\fi\fi}
\usepackage[latin1]{inputenc}

\ifx\dblatex@chaptersmark\@undefined\def\dblatex@chaptersmark#1{\markboth{\MakeUppercase{#1}}{}}\fi
\let\save@makeschapterhead\@makeschapterhead
\def\dblatex@makeschapterhead#1{\vspace*{-80pt}\save@makeschapterhead{#1}}
\def\@makeschapterhead#1{\dblatex@makeschapterhead{#1}\dblatex@chaptersmark{#1}}

			
\AtBeginDocument{\ifx\refname\@undefined\let\docbooktolatexbibname\bibname\def\docbooktolatexbibnamex{\bibname}\else\let\docbooktolatexbibname\refname\def\docbooktolatexbibnamex{\refname}\fi}
% Facilitate use of \cite with \label
\newcommand{\docbooktolatexbibaux}[2]{%
  \protected@write\@auxout{}{\string\global\string\@namedef{docbooktolatexcite@#1}{#2}}
}
% Provide support for bibliography `subsection' environments with titles
\newenvironment{docbooktolatexbibliography}[3]{
   \begingroup
   \let\save@@chapter\chapter
   \let\save@@section\section
   \let\save@@@mkboth\@mkboth
   \let\save@@bibname\bibname
   \let\save@@refname\refname
   \let\@mkboth\@gobbletwo
   \def\@tempa{#3}
   \def\@tempb{}
   \ifx\@tempa\@tempb
      \let\chapter\@gobbletwo
      \let\section\@gobbletwo
      \let\bibname\relax
   \else
      \let\chapter#2
      \let\section#2
      \let\bibname\@tempa
   \fi
   \let\refname\bibname
   \begin{thebibliography}{#1}
}{
   \end{thebibliography}
   \let\chapter\save@@chapter
   \let\section\save@@section
   \let\@mkboth\save@@@mkboth
   \let\bibname\save@@bibname
   \let\refname\save@@refname
   \endgroup
}

		
			
%\usepackage{cite}
%\renewcommand\citeleft{(}  % parentheses around list
%\renewcommand\citeright{)} % parentheses around list
\newcommand{\docbooktolatexcite}[2]{%
  \@ifundefined{docbooktolatexcite@#1}%
  {\cite{#1}}%
  {\def\@docbooktolatextemp{#2}\ifx\@docbooktolatextemp\@empty%
   \cite{\@nameuse{docbooktolatexcite@#1}}%
   \else\cite[#2]{\@nameuse{docbooktolatexcite@#1}}%
   \fi%
  }%
}
\newcommand{\docbooktolatexbackcite}[1]{%
  \ifx\Hy@backout\@undefined\else%
    \@ifundefined{docbooktolatexcite@#1}{%
      % emit warning?
    }{%
      \ifBR@verbose%
        \PackageInfo{backref}{back cite \string`#1\string' as \string`\@nameuse{docbooktolatexcite@#1}\string'}%
      \fi%
      \Hy@backout{\@nameuse{docbooktolatexcite@#1}}%
    }%
  \fi%
}

		
			
% --------------------------------------------
% A way to honour <footnoteref>s
% Blame j-devenish (at) users.sourceforge.net
% In any other LaTeX context, this would probably go into a style file.
\newcommand{\docbooktolatexusefootnoteref}[1]{\@ifundefined{@fn@label@#1}%
  {\hbox{\@textsuperscript{\normalfont ?}}%
    \@latex@warning{Footnote label `#1' was not defined}}%
  {\@nameuse{@fn@label@#1}}}
\newcommand{\docbooktolatexmakefootnoteref}[1]{%
  \protected@write\@auxout{}%
    {\global\string\@namedef{@fn@label@#1}{\@makefnmark}}%
  \@namedef{@fn@label@#1}{\hbox{\@textsuperscript{\normalfont ?}}}%
  }

		
			
% index labeling helper
\newif\ifdocbooktolatexprintindex\docbooktolatexprintindextrue
\let\dbtolatex@@theindex\theindex
\let\dbtolatex@@endtheindex\endtheindex
\def\theindex{\relax}
\def\endtheindex{\relax}
\newenvironment{dbtolatexindex}[1]
   {
\if@openright\cleardoublepage\else\clearpage\fi
\let\dbtolatex@@indexname\indexname
\def\dbtolatex@label{%
 \ifnum \c@secnumdepth >\m@ne \refstepcounter{chapter}\fi%
 \label{#1}\hypertarget{#1}{\dbtolatex@@indexname}%
 \global\docbooktolatexprintindexfalse}
\def\indexname{\ifdocbooktolatexprintindex\dbtolatex@label\else\dbtolatex@@indexname\fi}
\dbtolatex@@theindex
   }
   {
\dbtolatex@@endtheindex\let\indexname\dbtolatex@@indexname
   }

\newlength\saveparskip \newlength\saveparindent
\newlength\tempparskip \newlength\tempparindent

		
\def\docbooktolatexgobble{\expandafter\@gobble}
% Prevent multiple openings of the same aux file
% (happens when backref is used with multiple bibliography environments)
\ifx\AfterBeginDocument\undefined\let\AfterBeginDocument\AtBeginDocument\fi
\AfterBeginDocument{
   \let\latex@@starttoc\@starttoc
   \def\@starttoc#1{%
      \@ifundefined{docbooktolatex@aux#1}{%
         \global\@namedef{docbooktolatex@aux#1}{}%
         \latex@@starttoc{#1}%
      }{}
   }
}
% --------------------------------------------
% Hacks for honouring row/entry/@align
% (\hspace not effective when in paragraph mode)
% Naming convention for these macros is:
% 'docbooktolatex' 'align' {alignment-type} {position-within-entry}
% where r = right, l = left, c = centre
\newcommand{\docbooktolatex@align}[2]{\protect\ifvmode#1\else\ifx\LT@@tabarray\@undefined#2\else#1\fi\fi}
\newcommand{\docbooktolatexalignll}{\docbooktolatex@align{\raggedright}{}}
\newcommand{\docbooktolatexalignlr}{\docbooktolatex@align{}{\hspace*\fill}}
\newcommand{\docbooktolatexaligncl}{\docbooktolatex@align{\centering}{\hfill}}
\newcommand{\docbooktolatexaligncr}{\docbooktolatex@align{}{\hspace*\fill}}
\newcommand{\docbooktolatexalignrl}{\protect\ifvmode\raggedleft\else\hfill\fi}
\newcommand{\docbooktolatexalignrr}{}
\ifx\captionswapskip\@undefined\newcommand{\captionswapskip}{}\fi
\makeatother

\makeatletter
\def\Hy@Warning#1{}
\def\Hy@WarningNoLine#1{}
\makeatother
\title{\bfseries Theorems}
\author{John Doe, Ramon Casellas, and Some Corporation}
% --------------------------------------------
\makeindex
\makeglossary
% --------------------------------------------

\setcounter{tocdepth}{4}

\setcounter{secnumdepth}{4}
\begin{document}

\InputIfFileExists{title}{\typeout{WARNING: Using cover page title}}{\maketitle\pagestyle{fancy}
\thispagestyle{empty}}

% -------------------------------------------------------------
% Chapter Chapter1 
% ------------------------------------------------------------- 	
\chapter{Chapter1}
\label{chapter1}\hypertarget{chapter1}{}%

% ------------------------   
% Section 
\section{Section}
\label{sect11}\hypertarget{sect11}{}%

\begin{itemize}
%--- Item
\item 
Primary with secondary
\index{HTML!XML vs.}

%--- Item
\item 
{``}See{''}
\index{Hypertext Markup Language|textit{see} {HTML} }\index{quote characters!""\ensuremath{"|}!+\ensuremath{"|}}

%--- Item
\item 
Primary with secondary
\index{sorting!a}\index{sorting!c}

%--- Item
\item 
Primary with secondary and sortas.
\index{sorting!b@{{d (sorts like b)}}}

%--- Item
\item 
{\texttt} (should be ignored).
\index{samples!no bold and no emphasis}

%--- Item
\item 
{\texttt{{<?latex-\dbz{}index-\dbz{}style?>}}} with {\texttt} (should be ignored).
\index{samples!bold and no emphasis@{\textbf{bold and no emphasis}}}

%--- Item
\item 
{\texttt{{<?latex-\dbz{}index-\dbz{}style?>}}} with {\texttt} (should be honoured).
\index{samples!bold and emphasis@{\textbf{bold and emphasis}}}

%--- Item
\item 
sortas on primary with {\texttt{{<?latex-\dbz{}index-\dbz{}style?>}}} on secondary and {\texttt} on secondary (should be ignored).
\index{samples2@{{more samples}}!bold and no emphasis@{\textbf{bold and no emphasis}}}

%--- Item
\item 
triple sortas
\index{a@{{a}}!b@{\textbf{b}}!c@{{c}}}

%--- Item
\item 
another triple sortas
\index{a@{{a}}!b@{{b}}!d@{{d}}}

%--- Item
\item 
weird characters
\index{\textless{}\textbackslash \textasciitilde{}\textasciicircum{}\&""\ensuremath{"|}"!"@\textbraceleft{}\textbraceright{}\textbackslash \#\textbackslash \ \#\_\$\textbraceright{}\textasciicircum{}\%\textbraceleft{}\textgreater{}\textbackslash!\textless{}\textbackslash \textasciitilde{}\textasciicircum{}\&'\ensuremath{"|}"!"@\textbraceleft{}\textbraceright{}\textbackslash \#\textbackslash \ \#\_\$\textbraceright{}\textasciicircum{}\%\textbraceleft{}\textgreater{}\textbackslash@{{\textless{}\textbackslash \textasciitilde{}\textasciicircum{}\&'\ensuremath{"|}"!"@\textbraceleft{}\textbraceright{}\textbackslash \#\textbackslash \ \#\_\$\textbraceright{}\textasciicircum{}\%\textbraceleft{}\textgreater{}\textbackslash}}}\index{\textless{}!\textgreater{}!\textless{}\textbackslash \textasciitilde{}\textasciicircum{}\&""\ensuremath{"|}"!"@\textbraceleft{}\textbraceright{}\textbackslash \#\textbackslash \ \#\_\$\textbraceright{}\textasciicircum{}\%\textbraceleft{}\textgreater{}\textbackslash@{\textbf{\textless{}\textbackslash \textasciitilde{}\textasciicircum{}\&""\ensuremath{"|}"!"@\textbraceleft{}\textbraceright{}\textbackslash \#\textbackslash \ \#\_\$\textbraceright{}\textasciicircum{}\%\textbraceleft{}\textgreater{}\textbackslash}}}
\end{itemize}
\noindent 
% -------------------------------------------------------------
% Chapter Chapter2 
% ------------------------------------------------------------- 	
\chapter{Chapter2}
\label{chapter2}\hypertarget{chapter2}{}%

% ------------------------   
% Section 
\section{Section}
\label{sect12}\hypertarget{sect12}{}%

~ \index{HTML!XML vs.} \index{Hypertext Markup Language|textit{see} {HTML} } \index{SGML!HTML vs.}

% -------------------------------------------------------------
% Chapter Chapter3 
% ------------------------------------------------------------- 	
\chapter{Chapter3}
\label{chapter3}\hypertarget{chapter3}{}%

% ------------------------   
% Section 
\section{Section}
\label{sect13}\hypertarget{sect13}{}%

\index{HTML!XML vs.} \index{Hypertext Markup Language|textit{see} {HTML} } \index{SGML!HTML vs.}

% -------------------------------------------------------------
% Chapter Chapter4 
% ------------------------------------------------------------- 	
\chapter{Chapter4}
\label{chapter4}\hypertarget{chapter4}{}%

% ------------------------   
% Section 
\section{Section}
\label{sect14}\hypertarget{sect14}{}%

\index{HTML!XML vs.} \index{Hypertext Markup Language|textit{see} {HTML} } \index{SGML!HTML vs.}

% -------------------------------------------------------------
% Chapter Chapter5 
% ------------------------------------------------------------- 	
\chapter{Chapter5}
\label{chapter5}\hypertarget{chapter5}{}%

% ------------------------   
% Section 
\section{Section}
\label{sect15}\hypertarget{sect15}{}%

\index{HTML!XML vs.} \index{Hypertext Markup Language|textit{see} {HTML} } \index{SGML!HTML vs.}

% -------------------------------------------------------------
% Chapter Chapter6 
% ------------------------------------------------------------- 	
\chapter{Chapter6}
\label{chapter6}\hypertarget{chapter6}{}%

% ------------------------   
% Section 
\section{Section}
\label{sect16}\hypertarget{sect16}{}%

\index{HTML!XML vs.} \index{Hypertext Markup Language|textit{see} {HTML} } \index{SGML!HTML vs.}

% -------------------------------------------------------------
% Chapter Chapter7 
% ------------------------------------------------------------- 	
\chapter{Chapter7}
\label{chapter7}\hypertarget{chapter7}{}%

% ------------------------   
% Section 
\section{Section}
\label{sect17}\hypertarget{sect17}{}%

\index{HTML!XML vs.} \index{Hypertext Markup Language|textit{see} {HTML} } \index{SGML!HTML vs.}

% -------------------------------------------------------------
% Chapter Chapter8 
% ------------------------------------------------------------- 	
\chapter{Chapter8}
\label{chapter8}\hypertarget{chapter8}{}%

% ------------------------   
% Section 
\section{Section}
\label{sect18}\hypertarget{sect18}{}%

\index{HTML!XML vs.} \index{Hypertext Markup Language|textit{see} {HTML} } \index{SGML!HTML vs.}
\setlength\saveparskip\parskip
\setlength\saveparindent\parindent
\begin{dbtolatexindex}{index}
\setlength\tempparskip\parskip \setlength\tempparindent\parindent
\parskip\saveparskip \parindent\saveparindent
\noindent \indexspace
\parskip\tempparskip
\parindent\tempparindent
\makeatletter\@input@{\jobname.ind}\makeatother
\addcontentsline{toc}{chapter}{Index}
\end{dbtolatexindex}

\end{document}


\documentclass[pdf,frames,slideColor, rcas]{prosper}
\usepackage[latin1]{inputenc}
\usepackage{pstricks,pst-node,pst-text,pst-3d}
\usepackage{subfigure}
\usepackage{a4wide}
\usepackage{times}
\usepackage{fancyvrb}
\usepackage{amsmath,amsthm, amsfonts, amssymb, amsxtra,amsopn}
% Definition of new colors
\newrgbcolor{LemonChiffon}{1. 0.98 0.8}
\newrgbcolor{LightBlue}{0.68 0.85 0.9}
\hypersetup{pdfpagemode=FullScreen}
\makeatletter
%\newdimen\pst@dimz
%--------------------------------------------------SLIDES INFORMATION
\title{{\black The Slides Document Type}}
\subtitle{{\black Slides}}
\author{{\black }}

%\institution{%
%  Ecole Nationale Superieure des Telecommunications\\
%  46, Rue Barrault \\
%  75013 Paris, France}
%\email{casellas@infres.enst.fr}
%\slideCaption{GETENST Ramon Casellas}
\renewcommand{\slideparindent}{0mm}
%\raggedslides[0mm]
%\centerslidesfalse
\begin{document}
\maketitle
%--------------------------------------------------SLIDES INFORMATION
\title{{\black The Slides Document Type}}
\subtitle{{\black Slides}}
\author{{\black }}

%\institution{%
%  Ecole Nationale Superieure des Telecommunications\\
%  46, Rue Barrault \\
%  75013 Paris, France}
%\email{casellas@infres.enst.fr}
%\slideCaption{GETENST Ramon Casellas}
\renewcommand{\slideparindent}{0mm}
%\raggedslides[0mm]
%\centerslidesfalse

%---------------------------------------------------------------------- SLIDE 
\begin{slide}{Introduction}
\label{id2718407}

This is the introductory slide.

If you use foil groups (previously called sections), you can
have introductory slides before the first group.
\end{slide}
                                                                        

%---------------------------------------------------------------------- PART 
\part{Purpose and History            }
%---------------------------------------------------------------------- PART 
\label{id2718425}


%---------------------------------------------------------------------- SLIDE 
\begin{slide}{What Are Slides?}
\label{id2718436}
\begin{itemize}

	\item 
An XML presentation tool



	\item 
Suitable for HTML or PDF presentations



	\item 
Supported by Open Source software


\end{itemize}
\end{slide}

%---------------------------------------------------------------------- SLIDE 
\begin{slide}{Where Do They Come From?}
\label{id2718464}
\begin{itemize}

	\item 
Maintained by the \href{http://docbook.sourceforge.net/}{DocBook
Open Repository} Project at
\href{http://sourceforge.net/}{SourceForge}



	\item 
Customization layer of
\href{http://www.oasis-open.org/docbook/xml/simple/}{Simplified DocBook}


\end{itemize}
\end{slide}

%---------------------------------------------------------------------- SLIDE 
\begin{slide}{Why?}
\label{id2719177}
\begin{itemize}

	\item 
So Norm could give presentations



	\item 
So Norm could publish those presentations on the web



	\item 
So Norm could have {\em accessible} presentations
that didn't rely on the grotesque HTML output of some otherwise bloated, proprietary
tool



	\item 
So Norm could cut-and-paste from his DocBook articles and papers
directly into his slides



	\item 
Oh, let's face it: because it was there. Like the proverbial mountain.


\end{itemize}
\end{slide}
                                                                        

%---------------------------------------------------------------------- PART 
\part{Authoring            }
%---------------------------------------------------------------------- PART 
\label{id2719226}


%---------------------------------------------------------------------- SLIDE 
\begin{slide}{Minimal Presentation}
\label{id2719236}

The smallest possible presentation looks like this:

\begin{Verbatim}[]
<?xml version='1.0'?>
<!DOCTYPE slides PUBLIC "-//Norman Walsh//DTD Slides XML V3.0b1//EN"
                 "http://docbook.sourceforge.net/release/slides/3.0b1/slides.dtd">
<slides>
<slidesinfo>
<title>Presentation Title</title>
</slidesinfo>
<foil><title>Foil Title</title>
<para>Foil content</para>
</foil>
</slides>
\end{Verbatim}

Every presentation must contain at least one slide.
\end{slide}

%---------------------------------------------------------------------- SLIDE 
\begin{slide}{Presentation Metadata}
\label{id2719263}

Presentations usually have more metadata in the {\tt{slidesinfo}}
wrapper. Here's a typical example:

\begin{Verbatim}[]
<slidesinfo>
  <title>Supporting Localized Generated Text</title>
  <titleabbrev>Generated Text</titleabbrev>
  <author><firstname>Norman</firstname><surname>Walsh</surname></author>
  <pubdate>Sunday, 08 Apr 2001</pubdate>
  <confgroup>
    <conftitle>XSLTUK-01</conftitle>
    <confdates>08 Apr - 09 Apr 2001</confdates>
    <conftitle role="address">Keble College, Oxford, UK</conftitle>
    <confnum>1</confnum>
  </confgroup>
  <releaseinfo role="version">Version TEST</releaseinfo>
  <copyright><year>2001</year>
             <holder>Sun Microsystems, Inc.</holder></copyright>
</slidesinfo>
\end{Verbatim}
\end{slide}

%---------------------------------------------------------------------- SLIDE 
\begin{slide}{Presentation Content}
\label{id2719283}

It's common for individual slides to consist of a single
bulleted or numbered list. However, the full range of `block
level' Simplified DocBook elements are avialable.
\end{slide}
                                                                        

%---------------------------------------------------------------------- PART 
\part{Styling            }
%---------------------------------------------------------------------- PART 
\label{id2719315}


%---------------------------------------------------------------------- SLIDE 
\begin{slide}{HTML}
\label{id2719325}

There are a lot of HTML options. When you publish your
presentation on the web, it's probably best to use one of the simpler,
more accessible styles. For your actual live presentation, you may
want to choose something fancier.
\end{slide}

%---------------------------------------------------------------------- SLIDE 
\begin{slide}{Plain HTML}
\label{id2719340}
\begin{itemize}

	\item 
\href{../default/}{{\tt{default.xsl}}}
and
\href{../plain/}{{\tt{plain.xsl}}}
produce fairly simple presentations



	\item 
\href{../tables/}{{\tt{tables.xsl}}}
uses a table to show the navigation context (somewhat like the tabular
\href{http://docbook.sourceforge.net/}{Website} style)



	\item 
\href{../vslides/}{{\tt{vslides.xsl}}}
places navigation on the left side instead of the top and bottom



	\item 
\href{../w3c/}{{\tt{w3c.xsl}}}
produces slides that resemble the format used by the W3C for presentations


\end{itemize}
\end{slide}

%---------------------------------------------------------------------- SLIDE 
\begin{slide}{Fancy HTML}
\label{id2719423}
\begin{itemize}

	\item 
\href{../frames1/frames.html}{{\tt{frames.xsl}}}
uses frames. There are several options that you can apply:

\begin{itemize}

	\item 
\href{../frames2/frames.html}{overlay} uses CSS absolute positioning
to keep the navigation static on the pages (only works on recent browsers)



	\item 
\href{../frames3/frames.html}{multiframe} uses additional frames
to keep the navigation static on the pages



	\item 
\href{../frames4/frames.html}{dynamic.toc} uses JavaScript to keep
the table of contents and the current slide in sync (only works on recent browsers)



	\item 
\href{../frames5/frames.html}{active.toc} uses JavaScript to make
the table of context `active' so that you can open and close the foil
groups (only works on recent browsers)



	\item 
These toc styles can be combined with
\href{../frames6/frames.html}{overlay} or \href{../frames7/frames.html}{multiframe}


\end{itemize}

\end{itemize}
\end{slide}

%---------------------------------------------------------------------- SLIDE 
\begin{slide}{PDF}
\label{id2719531}

The {\tt{fo-plain.xsl}} stylesheet produces XSL Formatting
Objects that can subsequently be turned into PDF.
\end{slide}
                                                                        

%---------------------------------------------------------------------- PART 
\part{Presentation            }
%---------------------------------------------------------------------- PART 
\label{id2719548}


%---------------------------------------------------------------------- SLIDE 
\begin{slide}{Look And Feel}
\label{id2719559}

For HTML display, most of the actual
`look-and-feel' of the presentation is controlled by the CSS
stylesheet(s) used.
\end{slide}

%---------------------------------------------------------------------- SLIDE 
\begin{slide}{Presentation Tips}
\label{id2719574}
\begin{itemize}

	\item 
It's usually best if each slide is only one page (avoid scrolling).



	\item 
If you'll be giving your presentation with a projector, make sure you
know what resolution you'll have available and test your presentation at that resolution.



	\item 
Make your fonts bigger so the folks in the back of the room can read
your slides.



	\item 
Test the equipment before your presentation begins. Really.



	\item 
Speak more slowly. I always forget that one.


\end{itemize}
\end{slide}
                                                                        

%---------------------------------------------------------------------- PART 
\part{Conclusions            }
%---------------------------------------------------------------------- PART 
\label{id2719622}


%---------------------------------------------------------------------- SLIDE 
\begin{slide}{Conclusions}
\label{id2719628}

It's customary to have a conclusions slide.
\end{slide}

%---------------------------------------------------------------------- SLIDE 
\begin{slide}{References}
\label{id2719639}

References are a good idea too.
\end{slide}

%---------------------------------------------------------------------- SLIDE 
\begin{slide}{Q\&A}
\label{id2719651}

Any questions?
\end{slide}
\end{document}

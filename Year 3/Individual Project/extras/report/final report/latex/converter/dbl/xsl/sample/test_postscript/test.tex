% ----------------------- Preamble 
\documentclass[english,a4paper]{article}
\usepackage{amsmath,amsthm, amsfonts, amssymb, amsxtra,amsopn}
\usepackage{graphicx}
\usepackage{float}
\usepackage{times}
\usepackage[dvips]{hyperref}
\DeclareGraphicsExtensions{.eps}
\title{A short introduction to the XSL LaTeX Stylesheets in the framework of the DocBook DTD.}
\author{Ramon Casellas}
\begin{document}

\maketitle

% --------------------------------------------
% Abstract 
% --------------------------------------------
\begin{abstract}

 Most articles start with a nice abstract, which is usually left for 
the very last moment, and consists of several short sentences cut and
pasted basically from the introduction.

\end{abstract}


% ------------------------   
% Section 
\section{Introduction}
\label{id2718269}\hypertarget{id2718269}{}%

 This paper.... the purpose of this work... highlights...
Extensive simulations.... Numerical Results ...


% ------------------------   
% Section 
\section{Mathematical Model}
\label{id2718407}\hypertarget{id2718407}{}%

 The model proposed is as follows ... most ... simplicity...
analytical....


% ------------------------   
% Section 
\section{Numerical Results}
\label{id2718420}\hypertarget{id2718420}{}%

 To illustrate the main purpose of this paper ... 

% ------------------------   
% Section 
\section{Conclusion and future work}
\label{id2718431}\hypertarget{id2718431}{}%

This paper has given an insight ... 
% ------------------------------------------- 
%	
%  Bibliography
%	
% -------------------------------------------	
\bibliography{}
\begin{thebibliography}{WIDELABEL}

% -------------- biblioentry 
\bibitem[PimPum]{PimPum}
\emph{" Pim Pam Pum"} , James Pim and Robert Pum, Get a Life International Editions. ISBN 0-XS-xsxs-1, 2001. \label{PimPum}


\end{thebibliography}

% --------------------------------------------
% End of document
% --------------------------------------------
\end{document}
